%%%%%%%%%%%%%%%%%%%%%%% file template.tex %%%%%%%%%%%%%%%%%%%%%%%%%
%
% This is a general template file for the LaTeX package SVJour3
% for Springer journals.          Springer Heidelberg 2010/09/16
%
% Copy it to a new file with a new name and use it as the basis
% for your article. Delete % signs as needed.
%
% This template includes a few options for different layouts and
% content for various journals. Please consult a previous issue of
% your journal as needed.
%
%%%%%%%%%%%%%%%%%%%%%%%%%%%%%%%%%%%%%%%%%%%%%%%%%%%%%%%%%%%%%%%%%%%
%
% First comes an example EPS file -- just ignore it and
% proceed on the \documentclass line
% your LaTeX will extract the file if required

%
\RequirePackage{fix-cm}
%
\documentclass{svjour3}                     % onecolumn (standard format)
%\documentclass[smallcondensed]{svjour3}     % onecolumn (ditto)
%\documentclass[smallextended]{svjour3}       % onecolumn (second format)
%\documentclass[twocolumn]{svjour3}          % twocolumn
%
\smartqed  % flush right qed marks, e.g. at end of proof
%
\usepackage{moreverb,url}
\usepackage[colorlinks,bookmarksopen,bookmarksnumbered,citecolor=blue,urlcolor=blue]{hyperref}
\usepackage{graphicx}
\usepackage{natbib}
\usepackage{lineno}
\usepackage[misc]{ifsym}
\usepackage{bm}
\usepackage{amsmath}  
\usepackage{float}
\usepackage{multirow}
\usepackage{threeparttable}  
\usepackage{booktabs}
\usepackage{color}
\usepackage{amssymb}
\definecolor{revision_switch}{RGB}{255,0,0}
\graphicspath{{./fig/}}







%
% \usepackage{mathptmx}      % use Times fonts if available on your TeX system
%
% insert here the call for the packages your document requires
%\usepackage{latexsym}
% etc.
%
% please place your own definitions here and don't use \def but
% \newcommand{}{}
%
% Insert the name of "your journal" with
\journalname{Journal of Geodesy}
%
\begin{document}
\linenumbers
\title{GNSS-based displacement detection using Bayesian inference
%\thanks{Grants or other notes
%about the article that should go on the front page should be
%placed here. General acknowledgments should be placed at the end of the article.}
}


%\titlerunning{Short form of title}        % if too long for running head

\author{Nan Shen\textsuperscript{1}\and
		Liang Chen\textsuperscript{12*}\and        
		Ruizhi Chen\textsuperscript{12} %etc.
}

%\authorrunning{Short form of author list} % if too long for running head

\institute{
	\Letter Liang Chen\\
	{l.chen@whu.edu.cn}\\       
	\at
    {1} State Key Laboratory of Information Engineering in Surveying, Mapping and Remote Sensing, Wuhan University, 129 Luoyu Road, Wuhan 430079, China
    \at
    {2} Collaborative Innovation Center for Geospatial Technology,  129 Luoyu Road, Wuhan 430079, China
    \\
}

\date{Received: date }
% The correct dates will be entered by the editor


\maketitle

\begin{abstract}

\keywords{GNSS\and unmodeled error\and machine learning\and convolutional neural network\and wavelet coherence}
% \PACS{PACS code1 \and PACS code2 \and more}
% \subclass{MSC code1 \and MSC code2 \and more}
\end{abstract}

\section{Introduction}
\label{intro}
%全球定位系统系统的应用-》位移探测
%在自然灾害监测,或者工业测量等领域,GNSS被用于位移探测。
%传统测量仪器的弊端,GNSS位移探测的优势。
%GNSS在位移探测中扮演的角色,以及位移探测的要求
%目前人们对于基于GNSS位移探测的研究更过的关注与如何提高定位的精度,而忽略了如何从GNSS获取的坐标序列中识别并提取位移。
In recent decades, Global navigation satellite system (GNSS) has been widely used in navigation, surveying, and civil aviation as a positioning method\citep{teunissen2017springer}. 
Besides, as a displacement detection method, GNSS has been applied to natural disaster monitoring\citep{awange2012environmental}, industrial measurement\citep{pavasovic2011application}, and other fields. 
Compared with traditional displacement detection technology, GNSS-based displacement detection has many advantages, such as real-time, high accuracy, and weather independence\citep{shen2019a}. 
Displacement detection involves automatic identification of the time of displacement and determination of displacement. 
GNSS-based displacement detection is generally performed by the time-domain difference of GNSS positioning results. 
Previous studies on displacement detection based on GNSS pay more attention to how to improve the positioning accuracy. The research on how to identify and extract displacement from the coordinate sequence obtained by GNSS is limited.

%位移探测算法回顾,
%传统位移探测的弊端
%位移探测的短板1,关注与定位算法,忽略探测算法2,关注与longterm忽略shortterm
%随着在线监测系统的普及,急需对于单历元动态监测数据进行分析处理,今早发现识别和确定位移。
In order to evaluate the performance of single base station real time kinematic(RTK) and network RTK(NRTK) in displacement detection, the data of different observation duration were compared\citep{wang2011gps}.The results showed that NRTK has advantages in accuracy and robustness.
A displacement monitoring system was designed to evaluate NRTK for displacement detection\citep{GUMUS2019131}. The results showed that NRTK can achieve an accuracy of 4.7 mm and 7.9 mm in the horizontal and vertical directions respectively.
In \cite{csanliouglu2016landslide} and \cite{lytvyn2012real}, the feasibility of displacement detection using PPP has been studied.


\cite{gili2000using} was one of the first landslide monitoring studies using GNSS displacement detection technology, in which traditional measurement data are used as a comparison. The results showed that the accuracy of GPS measurement was 12 to 16 mm in the horizontal plane and 18 to 24 mm in the elevation.
A prototype of a low-cost GNSS has been developed for deep-seated landslide monitoring\citep{rs12203375}. The results of a nine-month filed monitoring period provided a detailed insight into the spatial and temporal pattern of deep-seated landslide surface movements, in which the displacement is obtained by the direct difference between the coordinates of the start and end periods.
In addition, GNSS is also combined with synthetic aperture radar technology for displacement detection, in which long-term GNSS measurements are used as references\citep{atanasova2018ground,fuhrmann2015estimation}.




%本文拟采取的措施弥补这些短板
%目前研究都是基于长周期时间序列进行分析,对于动态定位结果分析的研究还是有限。
%Bayes的优势,并且调研一下前人是否尝试同样的方法干过这类事情


%引入本文的研究内容和贡献


%引入文章结构


\section{Methodology}
\label{method}





\subsection{Displacement Detection}
The displacement time series is described as $\left[ {{x_1},{x_2}, \cdots {x_m}} \right]$,  and the corresponding time is denoted as $\left[ {{t_1},{t_2}, \cdots {t_m}} \right]$, where $m$ is the number of coordinate displacements. The time of displacement changes is expressed as $\left[ {{\tau _1},{\tau _2}, \cdots {\tau _n}} \right]$, which is a subset of $\left[ {{t_1},{t_2}, \cdots {t_m}} \right]$.  
The time corresponding to the $n$ change points satisfies the following relationship:$ {{\tau _1}<{\tau _2}<\cdots {\tau _{n-1}}<{\tau _n}} $.
The problem of displacement detection is defined as finding out the change points of displacement and giving the effective estimation of displacement.

\subsubsection{Bayesian Inference}
Bayesian inference is one of the most important skills in statistics. Bayesian inference deduces the posterior probability as the result of a priori probability and likelihood function. Bayesian inference calculates the posterior probability according to the Bayesian theorem

\begin{equation}\label{eq_bayesian_inference}
P({\bf{\theta }}\left| {\bf{x}} \right.) = \frac{{P({\bf{\theta }})P({\bf{x}}\left| {\bf{\theta }} \right.)}}{{P({\bf{x}})}}
\end{equation}
where $\bf{\theta }$ is the parameter to be estimated and $\bf{x}$ is the observation;$P({\bf{\theta }}\left| {\bf{x}} \right.)$ denotes the posterior probability; $P({\bf{\theta }})$ represents a priori probability, which refers to the probability obtained from previous experience and analysis;
$P({\bf{x}}\left| {\bf{\theta }} \right.)$ is the likelihood function, which represents the probability of $x$ when a priori is established. ${P({\bf{x}})}$ is the total likelihood, which is a constant value.

\subsubsection{Likelihood Function}
It is assumed that the displacement obtained by GNSS obeys Gaussian normal distribution. The displacement of each segment segmented by the change point obeys the Gaussian distribution of different mean and same variance, which is expressed as follows

\begin{equation}\label{eq_ts_cps}
x \sim \left\{ {\begin{array}{*{20}{r}}
	{N({\mu _0},\sigma ),}&{t < {\tau _1}}\\
	{N({\mu _1},\sigma ),}&{{\tau _1} \le t < {\tau _2}}\\
	\vdots &{}\\
	{\begin{array}{*{20}{c}}
		{N({\mu _n},\sigma ),}
		\end{array}}&{{\tau _{n - 1}} \le t < {\tau _n}}\\
	{N({\mu _{n + 1}},\sigma ),}&{{\tau _n} \le t}
	\end{array}} \right.
\end{equation}
where the $n$ change points divide the time series into $n+1$ segments, and the mean values of each segment are ${\mu _0},{\mu _1},\cdots,{\mu _n}$; $\sigma$ is the standard deviation of each normal distribution; Then the likelihood probability is expressed as follows

\begin{equation}\label{eq_likelihood}
P({\bf{x}}\left| {\bf{\theta }} \right.) = P({\bf{x}}\left| {{\mu_0},{\mu_1},} \right. \cdots {\mu_n},{\tau _1},{\tau _2}, \cdots ,{\tau _n},\sigma )
\end{equation}
where ${\bf{\theta }}=({{\mu_0},{\mu_1},} \cdots {\mu_n},{\tau _1},{\tau _2}, \cdots ,{\tau _n},\sigma) $; Assuming that the displacement observations are independent of each other, the likelihood probability can be expressed as

\begin{equation}\label{eq_likelihoodfunc}
P({\bf{x}}\left| {\bf{\theta }} \right.) = \prod\limits_{x \le {\tau _1}} {P(x\left| {{\mu_0},\sigma } \right.)} \prod\limits_{i = 1}^{n - 1} {\prod\limits_{{\tau _i} \le x < {\tau _{i + 1}}} {P(x\left| {{\mu_i},\sigma } \right.)} \prod\limits_{{\tau _n} \le x} {P(x\left| {{\mu_n},\sigma } \right.)} }.
\end{equation}
Combine \ref{eq_ts_cps} and \ref{eq_likelihoodfunc} to get the likelihood function as follows

\begin{equation}\label{eq_likelihoodfunc_detail}
P({\bf{x}}\left| {\bf{\theta }} \right.) = \frac{1}{{{{(2\pi {\sigma ^2})}^{m/2}}}}\exp \left\{ {\frac{1}{{ - 2{\sigma ^2}}}\left[ {\sum\limits_{x < {\tau _1}} {{{(x - {\mu_0})}^2}} {\rm{ + }}\sum\limits_{i = 1}^{n - 1} {\sum\limits_{{\tau _i} \le x < {\tau _{i + 1}}} {{{(x - {\mu_i})}^2}} } {\rm{ + }}\sum\limits_{{\tau _n} \le x} {{{(x - {\mu_n})}^2}} } \right]} \right\}
\end{equation}

\subsubsection{Prior Distribution}
Assuming that the parameters ${\bf{\tau }}=(\tau_1,\mu_2,\cdots,\tau_n)$, ${\bf{\mu }}=(\mu_0,\mu_1,\cdots,\mu_n)$ and $\sigma$ are independent of each other, then:

\begin{equation}\label{eq_bayesian_prior}
P({\bf{\theta }}) = P({\bf{\tau }},{\bf{\mu }},\sigma ) = P({\bf{\tau }})P({\bf{\mu }})P(\sigma )
\end{equation}
The posterior distribution consists of three parts, including the prior distribution of change point, mean value and standard deviation. It is assumed that the standard deviation $\sigma$ obeys the Gaussian normal distribution and is expressed as follows

\begin{equation}\label{eq_bayesian_prior_sigma}
\sigma  \sim N({\mu _\sigma },{\sigma _\sigma })
\end{equation}
where $\mu_{\sigma}$ and $\sigma_{\sigma}$ are determined according to the noise level of GNSS positioning.
For the mean value of the segments divided by the change point, it is assumed to obey the continuous uniform distribution, that is

\begin{equation}\label{eq_bayesian_prior_mu}
{\mu _i} \sim U({x_l },{x_u })(i = 0,1, \cdots ,n)
\end{equation}
where $x_l$ and $x_u$ represent the lower limit and  upper limit of the uniform distribution, which are determined by the mean value of the coordinate time series and the possible range of variation. Assuming that the mean values of each segment are independent of each other, then

\begin{equation}\label{eq_bayesian_prior_mu_sum}
P({\bf{\mu }}) = \prod\limits_{i = 0}^n {P({\mu _i})}
\end{equation}
For the change point ${\bf{\tau }}$, we only know that it is a subset of the corresponding time series.
It is assumed to obey the discrete uniform distribution. However, it should be noted that the lower limit of the distribution of the $i$th change point is determined by the $(i-1)$th change point, that is

\begin{equation}\label{eq_bayesian_prior_tau_basic}
P({\tau _i}\left| {{\tau _{i - 1}}} \right.) = \left\{ {\begin{array}{*{20}{r}}
	{DiscreteU({t_1},{t_m}),}&{i = 1}\\
	{DiscreteU({\tau _{i - 1}},{t_m}),}&{1 < i \le n}
	\end{array}} \right.
\end{equation}
The joint probability of all change points is expressed as

\begin{equation}\label{eq_bayesian_prior_tau_1}
P({\bf{\tau }}) = P({\tau _1},{\tau _2}, \cdots ,{\tau _{n - 1}},{\tau _n})
\end{equation}
According to the conditional probability at the change point ${\tau _1}$, the joint probability is expanded as

\begin{equation}\label{eq_bayesian_prior_tau_2}
P({\bf{\tau }}) = P({\tau _n},{\tau _{n - 1}}, \cdots ,{\tau _2}\left| {{\tau _1}} \right.)P({\tau _1})
\end{equation}
Further conditionally expand at the change point ${\tau _2}$ to get

\begin{equation}\label{eq_bayesian_prior_tau_3}
P({\bf{\tau }}) = P({\tau _n},{\tau _{n - 1}}, \cdots ,{\tau _3}\left| {{\tau _2},{\tau _1}} \right.)P(\left. {{\tau _2}} \right|{\tau _1})P({\tau _1})
\end{equation}
Since the conditional probability $P({\tau _n},{\tau _{n - 1}}, \cdots ,{\tau _3}\left| {{\tau _2},{\tau _1}} \right.)$ is directly determined by ${\tau _2}$, it has no direct relationship with ${\tau _1}$. So the above formula is reduced to

\begin{equation}\label{eq_bayesian_prior_tau_4}
P({\bf{\tau }}) = P({\tau _n},{\tau _{n - 1}}, \cdots ,{\tau _3}\left| {{\tau _2}} \right.)P(\left. {{\tau _2}} \right|{\tau _1})P({\tau _1})
\end{equation}
Continue to expand in a similar way, and finally get

\begin{equation}\label{eq_bayesian_prior_tau_5}
P({\bf{\tau }}) = P({\tau _n}\left| {{\tau _{n - 1}}} \right.) \cdots P({\tau _3}\left| {{\tau _2}} \right.)P(\left. {{\tau _2}} \right|{\tau _1})P({\tau _1})
\end{equation}
Through \ref{eq_bayesian_prior_tau_basic} and \ref{eq_bayesian_prior_tau_5}, the joint probability of all change points can be determined. Finally, the prior distribution of parameter ${\bf{\theta }}$ can be determined by formula \ref{eq_bayesian_prior}, \ref{eq_bayesian_prior_sigma}, \ref{eq_bayesian_prior_mu_sum}, \ref{eq_bayesian_prior_tau_5}.
\begin{figure}[H]
	\centering
	\includegraphics[width=8 cm]{v_simple_p_model}
	\caption{Daily displacement of experimental data obtained by the ionosphere-free PPP static model.}
	\label{fig_v_simple_p_model}
\end{figure} 

\subsubsection{Markov Chain Monte Carlo (MCMC)}
The posterior distribution of the parameters depends on the prior distribution and the likelihood function. Obtaining the posterior probability of parameters by direct integration or sum requires a lot of operations, which is difficult to achieve. In this work, the Markov chain Monte Carlo technique is used to approximate the integral or sum value. Monte Carlo sampling is to construct the integral or sum function as the expectation under a certain distribution, and then approximate the expectation by the corresponding average value\citep{goodfellow2016deep}.

\begin{equation}\label{eq_monte_carlo_1}
s =\sum\limits_x g(x)= \sum\limits_x {p(x)f(x) = {E_p}[f(x)]}
\end{equation}
\begin{equation}\label{eq_monte_carlo_2}
s =\int {g(x)}= \int {p(x)f(x)dx}  = {E_p}[f(x)]
\end{equation}
where $g(x)$ is the function to be integrated or summated; $p(x)$ is the probability distribution or probability density function of $x$. The integral or summation problem of $g(x)$ is transformed into the mathematical expectation problem of $f(x)$ under the distribution $p(x)$. This mathematical expectation can be approximated by sampling from $p(x)$.

\begin{equation}\label{eq_monte_carlo_3}
\hat s = \frac{1}{n}\sum\limits_{i = 1}^n {f({x_i})}
\end{equation}
where $x_i$ is the sample with the distribution of $p(x)$. As can be seen from the above, Monte Carlo method as a general sampling simulation sum or integration method, depends on sampling from the distribution of $p(x)$. However, it is not easy to directly sample from the target distribution of $p(x)$. At present, the most popular solution is to use Markov chain Monte Carlo sampling, which provides a general way to construct a sequence that converges to the target distribution. According to Markov chain, the probability of state transition at a certain time only depends on its previous state, which is expressed as follows

\begin{equation}\label{eq_markov_chain_1}
P({x_{n + 1}}\left| {{x_n},{x_{n - 1}}} \right.,{x_{n - 2}}, \cdots ,{x_2},{x_1}) = P({x_{n + 1}}\left| {{x_n}} \right.)
\end{equation}
If a Markov process is neither periodic nor irreducible, then the Markov process is ergodic, that is, each state appears with a certain probability. The Markov process of ergodic states has an important property: no matter what the initial state probability is, the Markov process of the ergodic state tends to be a stable distribution after enough state transitions.
That is, for the initial probability distribution is $\pi_0(x)$, the probability distribution after several rounds of state transition is as follows

\begin{equation}\label{eq_markov_chain_2}
\pi_{0}(x),\pi_{1}(x),\pi_{2}(x),\cdots \pi_{n}(x),\pi_{n+1}(x),\pi_{n+2}(x)\cdots
\end{equation}
There exists $n$ which satisfies the following conditions

\begin{equation}\label{eq_markov_chain_3}
\pi_{n}(x)=\pi_{n+1}(x)=\pi_{n+2}(x)\cdots
\end{equation}
This is the target distribution we want to construct. Based on the initial arbitrary simple probability distribution such as Gaussian distribution $\pi_{0}(x)$, the state value $x_0$ is obtained by sampling. Based on the probability distribution $\pi_{1}(x)=\pi_{0}(x)P(x\left| {{x_0}} \right.)$ sampling state value $x_1$, using the same method, execute $n$ times. Sample $(x_{n},x_{n+1},x_{n+2}\cdots)$ is the corresponding sample set of the target distribution. However, the implementation of MCMC also needs to consider other factors such as constructing the state transition matrix according to the target distribution, which is beyond the scope of this work. Two implementations of MCMC,  the Metropolis-Hasting sampler\citep{chib1995understanding} and the No-U-Turn sampler\citep{hoffman2014no}, are used to sample discrete random variables and continuous random variables, respectively.

\subsection{Relative real-time kinematic (RTK)}
RTK is widely used in structural health monitoring, landslide monitoring, and other fields due to its high positioning accuracy and real-time efficiency. The observation model, the state model, and the extended Kalman filtering process of RTK are introduced below. 
\subsubsection{Observation model}
The double-difference (DD) observation equation of relative positioning among base receiver, rover receiver, and satellites $j$, $k$ is defined in (\ref{eq_dd_obs})\citep{teunissen2017springer}.
\begin{figure*}
	\begin{equation}
	\left. \begin{array}{l}
	\Delta \nabla P_{{i_{br}}}^{jk} = \Delta \nabla \rho _{br}^{jk} - \Delta \nabla \delta I_{\Phi {i_{br}}}^{jk} + \Delta \nabla \delta _{tro{p_{br}}}^{jk} + \Delta \nabla \delta M_{{P_{{i_{br}}}}}^{jk} + \Delta \nabla \varepsilon _{{P_{{i_{br}}}}}^{jk}\\
	\Delta \nabla \Phi _{{i_{br}}}^{jk} = \Delta \nabla \rho _{br}^{jk} + \Delta \nabla \delta I_{{\Phi _{{i_{br}}}}}^{jk} + \Delta \nabla \delta _{tro{p_{br}}}^{jk} - {\lambda _i}\Delta \nabla N_{{i_{br}}}^{jk} + \Delta \nabla \delta M_{{\Phi _{{i_{br}}}}}^{jk} + \Delta \nabla \varepsilon _{{\Phi _{{i_{br}}}}}^{jk}
	\end{array} \right\}
	\label{eq_dd_obs}
	\end{equation}
\end{figure*}
In this observation equation, $\Delta \nabla $ denotes the DD operator, $i$ denotes the frequency; $b$ and $r$ denote base station and rover station respectively; $\Delta \nabla P_{{{i}_{br}}}^{jk}$ and $\Delta \nabla \Phi _{{{i}_{br}}}^{jk}$ denote pseudo-range and carrier-phase DD observations, respectively; $\Delta \nabla \rho _{br}^{jk}$ is the DD geometric distance between receivers and satellites; $\Delta \nabla \delta M_{{{P}_{{{i}_{br}}}}}^{jk}$ and $\Delta \nabla \delta M_{{{\Phi }_{{{i}_{br}}}}}^{jk}$ denote pseudo-range and carrier-phase multipath error, respectively; $\Delta \nabla \varepsilon _{{{P}_{{{i}_{br}}}}}^{jk}$ and $\Delta \nabla \varepsilon _{{{\Phi }_{{{i}_{br}}}}}^{jk}$ represent DD observation error of pseudo-range and carrier-phase, respectively; $\Delta \nabla \delta I_{\Phi {{i}_{br}}}^{jk}$is the DD ionospheric delay between receivers and satellites, and the ionospheric error has the same absolute value but the opposite sign on pseudo-range and carrier-phase observations; $\Delta \nabla \delta _{tro{{p}_{br}}}^{jk}$is the DD tropospheric delay between receivers and satellites. For the short baseline, the atmospheric error can be ignored, so the above DD observation equation is simplified as (\ref{eq_dd_obs_simplified}).
\begin{figure*}
	\begin{equation}
	\left. \begin{array}{l}
	\Delta \nabla P_{{i_{br}}}^{jk} = \Delta \nabla \rho _{br}^{jk} + \Delta \nabla \delta M_{{P_{{i_{br}}}}}^{jk} + \Delta \nabla \varepsilon _{{P_{{i_{br}}}}}^{jk}\\
	\Delta \nabla \Phi _{{i_{br}}}^{jk} = \Delta \nabla \rho _{br}^{jk} - {\lambda _i}\Delta \nabla N_{{i_{br}}}^{jk} + \Delta \nabla \delta M_{{\Phi _{{i_{br}}}}}^{jk} + \Delta \nabla \varepsilon _{{\Phi _{{i_{br}}}}}^{jk}\end{array} \right\}
	\label{eq_dd_obs_simplified}
	\end{equation}
	
\end{figure*}
By ignoring the influence of multipath, linearizing equation (\ref{eq_dd_obs_simplified}), we get
\begin{equation}
\label{eq_rtk_measure}
{\bf{y}} = {\bf{Hx}} + {\bf{\varepsilon }}
\end{equation}
where $\bf{y}$ is the observation minus the computed,  $\bf{H} $ is the linearized design matrix\citep{hofmann-wellenhof2007gnss}, $\bf{x}\triangleq {{[{{\mathbf{r}}_{r}},\Delta \nabla \mathbf{N}]}^{T}}$, is the state vector, which contains the rover station coordinate ${{\mathbf{r}}_{r}}$ and the DD integer ambiguity $\Delta \nabla \mathbf{N}$;  $\mathbf{\varepsilon } $ is the observation noise, and its covariance matrix is expressed as  ${{\mathbf{R}}_{\varepsilon }} $. After forming the observation equation, the ‘float’ solution of the parameter can be obtained by the standard least square method. To obtain high precision positioning results, it is necessary to fix the integer ambiguity. The 'float' solution and variance of integer ambiguity are expressed as  $\Delta \nabla \mathbf{\tilde{N}} $,  ${{\mathbf{Q}}_{\Delta \nabla \mathbf{\tilde{N}}}} $, respectively. Then the problem of ambiguity fixing can be described as (\ref{eq_lamda})\citep{teunissen1995the}.
\begin{figure*}
	\begin{equation}
	\Delta \nabla {\bf{\hat N}} = \mathop {\arg min}\limits_{\Delta \nabla {\bf{N}} \in {\bf{Z}}} ((\Delta \nabla {\bf{\tilde N}} - \Delta \nabla {\bf{\hat N}}){\bf{Q}}_{\Delta \nabla {\bf{\tilde N}}}^{ - 1}{(\Delta \nabla {\bf{\tilde N}} - \Delta \nabla {\bf{\hat N}})^T})
	\label{eq_lamda}
	\end{equation}
\end{figure*}
There are many methods referring to integer ambiguity resolution, among which LAMBDA (Least-squares Ambiguity Decorrelation Adjustment)\citep{teunissen1995the} method is the most widely used one. The integer ambiguity is obtained by the integer least square method. Then, the test process is executed to determine whether to accept or reject the integer ambiguity. If the test is passed, the fixed integer ambiguity can be substituted into (\ref{eq_rtk_measure}) to resolve the fixed solution, otherwise, the ‘float’ solution will be maintained.

\subsubsection{State model}
The state model is defined as follows (Takasu 2013)
\begin{equation}
{{\bf{x}}_k} = {\bf{F}}{{\bf{x}}_{k - 1}} + {\bf{w}}
\end{equation}
where ${\bf{F}} \buildrel \Delta \over = \left[ {\begin{array}{*{20}{c}}
	{{{\bf{I}}_{3 \times 3}}}&{\bf{0}}\\
	{\bf{0}}&{{{\bf{I}}_{(2n - 2)(2n - 2)}}}
	\end{array}} \right]$ , is the transition matrix from $k-1$ to $k$; $\mathbf{I}$ denotes the identity matrix; $n$ represents the number of satellites observed simultaneously by the base station and the rover station; Only GPS ${{L}_{1}}$ and ${{L}_{2}}$ frequency observations are considered, so $2n-2$ DD ambiguities are formed. $\mathbf{w}$ is the process noise, and its covariance matrix is expressed as ${{\mathbf{G}}_{w}}$. For kinematic positioning, the coordinate covariance component is set to be large to capture dynamic features, while the ambiguity covariance component is set to $\mathbf{0}$ to ensure the stability of positioning.
\subsubsection{Extended Kalman filtering}
Kalman filtering includes two processes: prediction update and measurement update\citep{grewal2001kalman}. The prediction process is as follows:
\begin{equation}
{{\bf{\hat x}}_{k|k - 1}} = {\bf{F}}{{\bf{\hat x}}_{k - 1}}
\end{equation}
\begin{equation}
{{\bf{Q}}_{k|k - 1}} = {\bf{F}}{{\bf{Q}}_{k - 1}}{{\bf{F}}^T} + {{\bf{G}}_w}(k)
\end{equation}
According to the definition of the state transition matrix above, it can be seen that the state value of the previous epoch remains unchanged. However, the coordinate covariance component becomes larger, while the ambiguity covariance component remains unchanged. The measurement update process is as follows:
\begin{equation}
{{\bf{v}}_k} = {\bf{y}}(k) - {\bf{H}}(k){{\bf{\hat x}}_{k|k - 1}}
\end{equation}
\begin{equation}
{{\bf{S}}_k} = {\bf{H}}(k){{\bf{Q}}_{k|k - 1}}{\bf{H}}{(k)^{\rm{T}}} + {{\bf{R}}_\varepsilon }(k)
\end{equation}
\begin{equation}
{\bf{K}} = {{\bf{Q}}_{k|k - 1}}{\bf{H}}_k^T{\bf{S}}_k^{ - 1}
\end{equation}
where ${{\mathbf{v}}_{k}}$ denotes the innovation vector, and ${{\mathbf{S}}_{k}}$ is the covariance matrix of the innovation vector; $\mathbf{K}$ is the state estimation gain matrix. The final state is estimated as follows:
\begin{equation}
{{\bf{\hat x}}_k} = {{\bf{\hat x}}_{k|k - 1}} + {\bf{K}}{{\bf{v}}_k}
\end{equation}
\begin{equation}
{{\bf{Q}}_k} = {{\bf{Q}}_{k|k - 1}} - {\bf{K}}{{\bf{S}}_k}{{\bf{K}}^T}
\end{equation}
where ${{\mathbf{\hat{x}}}_{k}}$ is the final estimate, and ${{\mathbf{Q}}_{k}}$ is the corresponding covariance matrix; ${{\mathbf{\hat{x}}}_{k}}$ is the only input of the proposed method.  

\section{Experiments and Results}
To verify the feasibility of the proposed method, we carried out a series of experiments, including simulation and field experiments. In the simulation experiment, the data affected by different noises are simulated. In the field experiment, several displacement changes were triggered manually, and the relative real-time kinematic model is used to process the GNSS observations.
\subsection{Simulation experiment}
In the experiment, we only simulate the position measurement. In this simulation data, Gaussian white noise with mean 0 mm and standard deviation 30 mm is added. At 1000 s, 2500 s, and 3200 s, the displacements of -50 mm, -50 mm, and 100 mm are added, respectively. The simulation results are shown in Fig. \ref{fig_v_sim_simple_gauss_ts}.

\begin{figure}[htbp]
	\centering
	\includegraphics[width=10 cm]{v_sim_simple_gauss_ts}
	\caption{Simulation data with multiple displacement change points.}
	\label{fig_v_sim_simple_gauss_ts}
\end{figure} 
As shown in Fig. \ref{fig_v_sim_simple_gauss_ts}, the whole time series is divided into four segments by three displacement change points. Using the model designed above, Bayesian inference is realized by MCMC sampling. The histogram of posterior estimation of displacement change points is shown in Fig. \ref{fig_v_tau_sim_simple}.

\begin{figure}[htbp]
	\centering
	\includegraphics[width=10 cm]{v_tau_sim_simple}
	\caption{Histogram of posterior estimation of displacement change points in the simulation experiment.}
	\label{fig_v_tau_sim_simple}
\end{figure} 
As can be seen from Fig. \ref{fig_v_tau_sim_simple}, all the designed displacement change points are identified. $\tau_3$ is the easiest to be identified, mainly due to the largest displacement amplitude. Due to the small displacement amplitude and fewer sample points for Bayesian estimation on the right, the estimation accuracy at $\tau_2$ is worse than the other two points.

\begin{figure}[htbp]
	\centering
	\includegraphics[width=10 cm]{v_mu_sim_simple}
	\caption{Histogram of posterior estimation of the mean in the simulation experiment.}
	\label{fig_v_mu_sim_simple}
\end{figure} 

\begin{figure}[htbp]
	\centering
	\includegraphics[width=10 cm]{v_sigma_sim_simple}
	\caption{Histogram of posterior estimation of the measurement noise in the simulation experiment.}
	\label{fig_v_sigma_sim_simple}
\end{figure} 



\subsection{Field experiment}
The experiment was carried out on the roof of a building on the campus of Wuhan University. In the experiment, the equipment deployment is shown in Fig. \ref{fig_field_exp_config}. A hard plank is pressed with large stones to ensure that the plank remains as fixed as possible during moving the object hanging on the plank. One receiver antenna is fixed on the plank as a rover; the other receiver antenna is fixed on a tripod beside it as the base station. Both antennas are connected to BD992 OEM boards on the table, and the OEM boards are connected to the laptops for data collection. As shown in subgraphs (c) and (d), the height of the antenna is measured with a tape before and after each movement of the object hanging on the plank. Besides, markers are made on the ground and on the antenna respectively to ensure that the same position is referenced for each measurement. The sampling frequency of the GNSS receiver is set to 1 Hz. The whole experiment lasted for nearly one hour, during which the object hanging on the plank was moved several times manually.

\begin{table*}[h!t]
	\centering
	%	\fontsize{6.5}{8}\selectfont
	\begin{threeparttable}
		\caption{Measured height before and after each movement of the object hanging on the plank, unit (mm).}
		\label{tab_measured_height}
		\begin{tabular}{ccccccccc}
			\toprule
			&&&\textbf{1}&\textbf{2}&\textbf{3}&\textbf{4}&\textbf{5}&\textbf{mean}\cr
			\midrule
			\multirow{3}*{Step 1} 
			& & ${h_0}$    &371.0&370.9&370.9&371.1&370.9&370.96\\
			& & ${h_1}$    &341.0&341.1&341.5&341.9&341.0&341.30\\
			& & $\Delta h$ & 30.0& 29.8& 29.4& 29.2& 29.9&29.66\\
			\hline
			\multirow{3}*{Step 2} 
			& & ${h_0}$    &340.5&340.1&340.0&340.0&340.1&340.14\\
			& & ${h_1}$    &313.0&312.9&312.9&313.0&313.0&312.96\\
			& & $\Delta h$ &27.5& 27.2& 27.1& 27.0& 27.1& 27.18\\
			\bottomrule
		\end{tabular}
	\end{threeparttable}
\end{table*}

\begin{figure}[htbp]
	\centering
	\includegraphics[width=10 cm]{v_bd9_3_up_ts}
	\caption{Daily displacement of experimental data obtained by the ionosphere-free PPP static model.}
	\label{fig_v_bd9_3_up_ts}
\end{figure} 

\begin{figure}[htbp]
	\centering
	\includegraphics[width=10 cm]{v_bd9_3_up_tau}
	\caption{Daily displacement of experimental data obtained by the ionosphere-free PPP static model.}
	\label{fig_v_bd9_3_up_tau}
\end{figure} 

\begin{figure}[htbp]
	\centering
	\includegraphics[width=10 cm]{v_bd9_3_up_sigma}
	\caption{Daily displacement of experimental data obtained by the ionosphere-free PPP static model.}
	\label{fig_v_bd9_3_up_sigma}
\end{figure} 
\begin{figure}[htbp]
	\centering
	\includegraphics[width=10 cm]{v_bd9_3_up_mu}
	\caption{Daily displacement of experimental data obtained by the ionosphere-free PPP static model.}
	\label{fig_v_bd9_3_up_mu}
\end{figure} 

\begin{table*}[h!t]
	\centering
	%	\fontsize{6.5}{8}\selectfont
	\begin{threeparttable}
		\caption{Measured height before and after each movement of the object hanging on the plank, unit (m).}
		\label{tab_measured_height}
		\begin{tabular}{ccccccccc}
			\toprule
			&&&\textbf{1}&\textbf{2}&\textbf{3}&\textbf{4}&\textbf{5}&\textbf{mean}\cr
			\midrule
			\multirow{3}*{Step 1} 
			& & ${h_0}$    &368.5&	368.9&	369.0&	368.9&	369.0&	368.9\\
			& & ${h_1}$    &348.1&	348.0&	348.0&	348.2&	348.3&	348.1\\
			& & $\Delta h$ & 20.4&	 20.9&	 21.0&	 20.7&	 20.7&	 20.7\\
			\hline               
			\multirow{3}*{Step 2}
			& & ${h_0}$    &347.5&	347.3&	347.6&	347.5&	347.4&	347.5\\
			& & ${h_1}$    &326.1&	326.3&	326.1&	326.0&	326.1&	326.1\\
			& & $\Delta h$ & 21.4&	 21.0&	 21.5&	 21.5&	 21.3&	 21.3\\
			\hline               
			\multirow{3}*{Step 3}
			& & ${h_0}$    &325.0&	325.1&	324.9&	325.0&	325.2&	325.0\\
			& & ${h_1}$    &305.5&	305.3&	305.4&	305.5&	305.6&	305.5\\
			& & $\Delta h$ & 19.5&	 19.8&	 19.5&	 19.5&	 19.6&	 19.6\\
			\bottomrule
		\end{tabular}
	\end{threeparttable}
\end{table*}
\begin{figure}[htbp]
	\centering
	\includegraphics[width=10 cm]{v_bd9_4_up_ts}
	\caption{Daily displacement of experimental data obtained by the ionosphere-free PPP static model.}
	\label{fig_v_bd9_4_up_ts}
\end{figure} 

\section{Discussion}


\section{Conclusions}


\begin{acknowledgements}
This research was funded by the National Key Research and Development Programs 2018YFB0505400, the Natural Science Fund of Hubei Province 2018CFA007.
\end{acknowledgements}

%
\section*{Conflict of Interest}
The authors declare that they have no conflict of interest.
\section*{Author Contributions}
Conceptualization: Nan Shen, Liang Chen; Methodology: Nan Shen; Writing - original draft preparation: Nan Shen; Writing - review and editing:Ruizhi Chen; Supervision: Liang Chen.
 \section*{Data Availability}


% BibTeX users please use one of
\bibliographystyle{spbasic}      % basic style, author-year citations
%\bibliographystyle{spmpsci}      % mathematics and physical sciences
%\bibliographystyle{spphys}       % APS-like style for physics
\bibliography{ref}


\end{document}


